
%% bare_conf.tex
%% V1.3
%% 2007/01/11
%% by Michael Shell
%% See:
%% http://www.michaelshell.org/
%% for current contact information.
%%
%% This is a skeleton file demonstrating the use of IEEEtran.cls
%% (requires IEEEtran.cls version 1.7 or later) with an IEEE conference paper.
%%
%% Support sites:
%% http://www.michaelshell.org/tex/ieeetran/
%% http://www.ctan.org/tex-archive/macros/latex/contrib/IEEEtran/
%% and
%% http://www.ieee.org/

\documentclass[conference]{IEEEtran}



%stam my packages

\usepackage{graphicx}
\usepackage[xetex,colorlinks=true,urlcolor = black, 
linkcolor=black,citecolor=black]{hyperref}





% correct bad hyphenation here
\hyphenation{op-tical net-works semi-conduc-tor}


\begin{document}
%
% paper title
% can use linebreaks \\ within to get better formatting as desired
\title{ECSE 548 Project: 8-bit Booth Multiplier\\Design specifications}

%\vspace{-20pt}
\author{\IEEEauthorblockN{Marco Kassis,
Aryan Mojtahedi,
Dimitrios Stamoulis and
Louis-Charles Trudeau}
\IEEEauthorblockA{Department of Electrical and Computer Engineering\\
McGill University, Montreal, Canada}
~\\~\\
}





% use for special paper notices
%\IEEEspecialpapernotice{(Invited Paper)}


% make the title area
\maketitle


%\begin{abstract}
%\boldmath
%The abstract goes here.
%\end{abstract}
% IEEEtran.cls defaults to using nonbold math in the Abstract.
% This preserves the distinction between vectors and scalars. However,
% if the conference you are submitting to favors bold math in the abstract,
% then you can use LaTeX's standard command \boldmath at the very start
% of the abstract to achieve this. Many IEEE journals/conferences frown on
% math in the abstract anyway.

% no keywords




% For peer review papers, you can put extra information on the cover
% page as needed:
% \ifCLASSOPTIONpeerreview
% \begin{center} \bfseries EDICS Category: 3-BBND \end{center}
% \fi
%
% For peerreview papers, this IEEEtran command inserts a page break and
% creates the second title. It will be ignored for other modes.
\IEEEpeerreviewmaketitle

%~\\
%\vspace{+10pt}
\section{Introduction to Booth's algorithm}
The traditional way to compute a signed $M\times N$ multiplication is performed by properly
adding M sign-extended and shifted partial products. A signed multiplication therefore implies
the summation of M partial products and is expected to be slow when M contains many bits.
Each columns sums M bits and gives one or more carry bits to the following, leaving the
multiplication result's MSB waiting for $2\times M-2$ carry signals to be generated. In order to increase
the multiplier's speed and reduce its area, a modified Booth algorithm can be used. For a Radix-4 implementation, this reduces the amount of partial products by a factor of 2.

The Booth algorithm is quite simple; it encodes the multiplier N in overlapping groups of
three bits to generate three control signals, SINGLE, DOUBLE and NEGATE. The decoder can
compute the partial product according to the latter in five different ways:

\begin{enumerate}
\item PP=0's (multiplication by 0)
\item PP=M (multiplication by 1)
\item PP=-M (multiplication by -1)
\item PP=2M (multiplication by 2)
\item PP=-2M (multiplication by -2)
\end{enumerate}


All these partial products are then sign extended appropriately and shifted left by 2 from
each other. The two last stages of the multiplication compute the sum of these partial products.
The result is (M+N) bits wide.

%\hfill October 09, 2013



\section{Booth Multiplier Architecture}

Since our application is an 8x8 Booth multiplier, we have broken down the design in four
distinct parts: 1) the encoder, 2) the selector (decoder), 3) the compressor tree and 4) the final
addition. A schematic was created for each wordslice module in Electric and was tested for
DRCs. They were then simulated in ModelSim for verification of correctness.


\subsection{Radix-4 Modified Booth Encoder}

For an 8x8 Booth multiplier, the wordslice encoder is made of 4 bitslice encoder. The
latter encodes the multiplier using N[7:5], N[5:3], N[3:1] and N[1:0]\&0 to generate SINGLE[3:0],
DOUBLE[3:0] and NEGATE[3:0].



\subsection{Booth Selector}

The Booth selector uses the three encoded vectors to generate four different 9bits wide
partial products. Its bitslice design is quite simple; you either shift left by one (multiply by 2) or
invert the bit or both. The wordslice's first subtlety is to remember to add `1' when you negate
the signal, to satisfy the two's complement representation. For that matter, 9 half-adders were
used to sum the NEGATE signal (0 when positive, 1 when negative; propagate). The second
subtlety was in the sign generation. The selector and half-adder would cover almost every
possible case, except when Y= -128 and PP= -2M. The correct answer is 100000000 and sign=0.
A simple 3-stage NAND8 reducer was used to cover this special case.



\subsection{Partial Products Compressor Tree}

The compressor tree uses thirteen [4:2] compressors optimized for area to
sum the columns. Each compressor has five inputs: A, B, C, D and $C_{int-1}$ and three outputs: Sum,
$C_{int}$ and $C_{ext}$. The internal carry $C_{int}$ is propagated to the next column and the external carry $C_{ext}$ is
given to the final addition, thus balancing the propagation delay.


\subsection{Final Addition}

The final addition is done using a 13bits carry-ripple adder. This type of adder is
optimized for area and can easily be implemented in layout, but has the longest propagation
delay of all adders.


\section{8x8 Booth Multiplier Simulation}

An additional simulation was done in Quartus II to test the overall functionality of our Booth
multiplier. The integrated ModelSim simulator was found to be very useful to quickly generate
stimuli vectors. Our module’s result is compared to the output of a generic 8x8 multiplier using a
XOR-tree and is tested extensively.



% An example of a floating figure using the graphicx package.
% Note that \label must occur AFTER (or within) \caption.
% For figures, \caption should occur after the \includegraphics.
% Note that IEEEtran v1.7 and later has special internal code that
% is designed to preserve the operation of \label within \caption
% even when the captionsoff option is in effect. However, because
% of issues like this, it may be the safest practice to put all your
% \label just after \caption rather than within \caption{}.
%
% Reminder: the "draftcls" or "draftclsnofoot", not "draft", class
% option should be used if it is desired that the figures are to be
% displayed while in draft mode.
%
%\begin{figure}[!t]
%\centering
%\includegraphics[width=2.5in]{myfigure}
% where an .eps filename suffix will be assumed under latex, 
% and a .pdf suffix will be assumed for pdflatex; or what has been declared
% via \DeclareGraphicsExtensions.
%\caption{Simulation Results}
%\label{fig_sim}
%\end{figure}

% Note that IEEE typically puts floats only at the top, even when this
% results in a large percentage of a column being occupied by floats.


% An example of a double column floating figure using two subfigures.
% (The subfig.sty package must be loaded for this to work.)
% The subfigure \label commands are set within each subfloat command, the
% \label for the overall figure must come after \caption.
% \hfil must be used as a separator to get equal spacing.
% The subfigure.sty package works much the same way, except \subfigure is
% used instead of \subfloat.
%
%\begin{figure*}[!t]
%\centerline{\subfloat[Case I]\includegraphics[width=2.5in]{subfigcase1}%
%\label{fig_first_case}}
%\hfil
%\subfloat[Case II]{\includegraphics[width=2.5in]{subfigcase2}%
%\label{fig_second_case}}}
%\caption{Simulation results}
%\label{fig_sim}
%\end{figure*}
%
% Note that often IEEE papers with subfigures do not employ subfigure
% captions (using the optional argument to \subfloat), but instead will
% reference/describe all of them (a), (b), etc., within the main caption.


% An example of a floating table. Note that, for IEEE style tables, the 
% \caption command should come BEFORE the table. Table text will default to
% \footnotesize as IEEE normally uses this smaller font for tables.
% The \label must come after \caption as always.
%
%\begin{table}[!t]
%% increase table row spacing, adjust to taste
%\renewcommand{\arraystretch}{1.3}
% if using array.sty, it might be a good idea to tweak the value of
% \extrarowheight as needed to properly center the text within the cells
%\caption{An Example of a Table}
%\label{table_example}
%\centering
%% Some packages, such as MDW tools, offer better commands for making tables
%% than the plain LaTeX2e tabular which is used here.
%\begin{tabular}{|c||c|}
%\hline
%One & Two\\
%\hline
%Three & Four\\
%\hline
%\end{tabular}
%\end{table}


% Note that IEEE does not put floats in the very first column - or typically
% anywhere on the first page for that matter. Also, in-text middle ("here")
% positioning is not used. Most IEEE journals/conferences use top floats
% exclusively. Note that, LaTeX2e, unlike IEEE journals/conferences, places
% footnotes above bottom floats. This can be corrected via the \fnbelowfloat
% command of the stfloats package.



%\section{Conclusion}
%The conclusion goes here.




% conference papers do not normally have an appendix


% use section* for acknowledgement
%\section*{Acknowledgment}
%The authors would like to thank...





% trigger a \newpage just before the given reference
% number - used to balance the columns on the last page
% adjust value as needed - may need to be readjusted if
% the document is modified later
%\IEEEtriggeratref{8}
% The "triggered" command can be changed if desired:
%\IEEEtriggercmd{\enlargethispage{-5in}}

% references section

% can use a bibliography generated by BibTeX as a .bbl file
% BibTeX documentation can be easily obtained at:
% http://www.ctan.org/tex-archive/biblio/bibtex/contrib/doc/
% The IEEEtran BibTeX style support page is at:
% http://www.michaelshell.org/tex/ieeetran/bibtex/
%\bibliographystyle{IEEEtran}
% argument is your BibTeX string definitions and bibliography database(s)
%\bibliography{IEEEabrv,../bib/paper}
%
% <OR> manually copy in the resultant .bbl file
% set second argument of \begin to the number of references
% (used to reserve space for the reference number labels box)





% that's all folks
\end{document}


